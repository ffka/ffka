\documentclass[12pt,a4paper,titlepage]{scrartcl}
\usepackage[utf8]{inputenc}
\usepackage[german]{babel}
\usepackage{amsmath}
\usepackage{amsfonts}
\usepackage{amssymb}
\usepackage{enumerate}
\usepackage{listings}
\usepackage{url}
\usepackage{MnSymbol}
\usepackage{eurosym}
\usepackage{color}
\definecolor{grey}{rgb}{0.9,0.9,0.9}


\usepackage{graphicx}               % erlaubt einfügen von Bildern
\usepackage{float}                  %Bilder Fixieren                            
\usepackage{caption}                % Für Captions in Minipage (für bilder alternativ zu \caption ->\captionof verwenden
\graphicspath{{./images/}}          % suche nach Bildern im angegebenen Ordner
\usepackage[printonlyused]{acronym} %ermöglicht das Arbeiten mit Abk. und Verzeichniss
\usepackage{placeins}               % für \FloatBarrier damit figures auch wirklich genau da stehen bleiben so sie sind... Nicht schön aber ich glaub der Holzhammer muss hier sein... 
\usepackage{cite}                                                               
\def\BibTeX{{\rm B\kern-.05em{\sc i\kern-.025em b}\kern-.08em                   
   T\kern-.1667em\lower.7ex\hbox{E}\kern-.125emX}}
   
\title{Beitragsordnung Verein zur Förderung freier Netze Region Mittlerer Oberrhein e.V.}
\lstset{prebreak=\raisebox{0ex}[0ex][0ex]
        {\ensuremath{\rhookswarrow}}}
\lstset{postbreak=\raisebox{0ex}[0ex][0ex]
        {\ensuremath{\rcurvearrowse\space}}}
\lstset{
	breaklines=true,
	breakatwhitespace=false,
	backgroundcolor=\color{grey},
}
\renewcommand*\thesection{\S~\arabic{section}}
\begin{document}
\maketitle
\pagenumbering{Roman}
\thispagestyle{empty}
\newpage
\pagenumbering{arabic}
\setcounter{page}{1}

\section{Höhe der Beiträge}
\begin{enumerate}
\item Der Mitgliedsbeitrag für eine ordentliche Mitgliedschaft beträgt mindestens 60 \euro{} pro Jahr.
\item Der Mitgliedsbeitrag für Fördermitglieder beträgt für natürliche Personen mindestens 60 \euro{} pro Jahr und für juristsche Personen mindestens 300 \euro{} pro Jahr.
\item Ehrenmitglieder sind laut Satzung von der Beitragszahlung befreit.
\item Im begründeten Einzelfall können für ein Mitglied durch Vorstandsbeschluss von der Beitragsordnung abweichende Regelungen getroffen werden.
\end{enumerate}
	
\section{Aufnahmegebühren, Kündigung und Erstattungen}
\begin{enumerate}
\item Aufnahmegebühren werden nicht erhoben.
\item Eine Erstattung von Mitgliedsbeiträgen findet nicht statt.
\end{enumerate}


\section{Fälligkeit und Zahlungsweise}
\begin{enumerate}
\item Der Mitgliedsbeitrag ist zum 01. Januar eines jeden Jahres fällig.
\item Der Mitgliedsbeitrag für das laufende Jahr wird anteilig für jeden noch nicht begonnenen Monat berechnet und ist sofort fällig.
\item Der Mitgliedsbeitrag kann per Lastschriftverfahren vom Verein eingezogen, durch Banküberweisung an den Verein oder durch Barzahlung an den Schatzmeister beglichen werden.
\item Für abgelehnte Lastschriften oder Rücklastschriften werden die anfallenden Gebühren in Rechnung gestellt.
\end{enumerate}
\end{document}