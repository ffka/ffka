\documentclass[12pt,a4paper,titlepage]{scrartcl}
\usepackage[utf8]{inputenc}
\usepackage[german]{babel}
\usepackage{amsmath}
\usepackage{amsfonts}
\usepackage{amssymb}
\usepackage{enumerate}
\usepackage{listings}
\usepackage{url}
\usepackage{MnSymbol}
\usepackage{color}
\definecolor{grey}{rgb}{0.9,0.9,0.9}


\usepackage{graphicx}               % erlaubt einfügen von Bildern
\usepackage{float}                  %Bilder Fixieren                            
\usepackage{caption}                % Für Captions in Minipage (für bilder alternativ zu \caption ->\captionof verwenden
\graphicspath{{./images/}}          % suche nach Bildern im angegebenen Ordner
\usepackage[printonlyused]{acronym} %ermöglicht das Arbeiten mit Abk. und Verzeichniss
\usepackage{placeins}               % für \FloatBarrier damit figures auch wirklich genau da stehen bleiben so sie sind... Nicht schön aber ich glaub der Holzhammer muss hier sein... 
\usepackage{cite}                                                               
\def\BibTeX{{\rm B\kern-.05em{\sc i\kern-.025em b}\kern-.08em                   
   T\kern-.1667em\lower.7ex\hbox{E}\kern-.125emX}}
   
\title{Satzung Freifunk Karlsruhe e.V.}
\lstset{prebreak=\raisebox{0ex}[0ex][0ex]
        {\ensuremath{\rhookswarrow}}}
\lstset{postbreak=\raisebox{0ex}[0ex][0ex]
        {\ensuremath{\rcurvearrowse\space}}}
\lstset{
	breaklines=true,
	breakatwhitespace=false,
	backgroundcolor=\color{grey},
}
\renewcommand*\thesection{\S~\arabic{section}}
\begin{document}
\maketitle
\pagenumbering{Roman}
\thispagestyle{empty}
\newpage
\pagenumbering{arabic}
\setcounter{page}{1}

\section{Name und Sitz des Vereins}
\begin{enumerate}
\item Der Verein führt den Namen ''Freifunk Karlsruhe'' und soll in das Vereinsregister des Amtsgerichts Mannheim eingetragen werden. Nach der Eintragung führt er den Zusatz ''e.V.''.
\item Der Sitz des Vereins ist Karlsruhe.
\item Das Geschäftsjahr des Vereins ist das Kalenderjahr. 
\item Weitere Details und Verfahrensabläufe werden in einer, der Satzung untergeordneten, Geschäftsordnung (GO) geregelt.
\end{enumerate}

\section{Zweck des Vereins, Gemeinnützigkeit}
\begin{enumerate}
\item Zweck des Vereins ist die Erforschung, Anwendung und Verbreitung freier 
Netzwerktechnologien, Verbreitung und Vermittlung von Wissen über Funk und 
Netzwerktechnologien,  die Förderung von Volks- und Berufsbildung sowie der Erziehung der Jugend zum verantwortungsvollen Umgang mit neuen Medien (Medienkompetenz).
\item Der Satzungszweck wird insbesondere verwirklicht durch:
\begin{itemize}
\item die Förderung der Bildung und Forschung bezüglich moderner Kommunikationsnetze, insbesondere durch das Internet und durch Vorträge, Veranstaltungen, Vorführungen und Publikationen
\item die Förderung und Unterstützung des Zugangs zu Informationstechnologie für sozial benachteiligte Personen
\item die Schaffung experimenteller Kommunikations- und Infrastrukturen sowie Bürgerdatennetzen
\item Kulturelle, technologische und soziale Bildungs- und Forschungsobjekte, die Veranstaltung regionaler, nationaler und internationaler Kongresse, Treffen und Konferenzen, sowie die Teilnahme der Mitglieder
\end{itemize}
\item Der Verein ist politisch und konfessionell unabhängig.
\item Der Verein ist selbstlos tätig und verfolgt nicht in erster Linie eigenwirtschaftliche Zwecke. Er verfolgt ausschließlich und unmittelbar gemeinnützige Zwecke im Sinne des Abschnitts ''Steuerbegünstigte Zwecke'' der Abgabenordnung (AO).
\item Die Mittel des Vereins dürfen nur für satzungsgemäße Zwecke verwendet werden. Es darf keine Person durch Ausgaben, die dem Vereinszweck fremd sind oder durch unverhältnismäßig hohe Vergütungen begünstigt werden.
\item Die Mitglieder erhalten keine Zuwendungen aus den Mitteln des Vereins.
\end{enumerate}


\section{Erwerb der Mitgliedschaft}
Mitglieder können natürliche und juristische Personen, z. B. Firmen, Vereine, Verbände und Behörden werden, die gewillt sind, die gemeinnützigen Ziele des Vereins zu fördern und diesen in der Durchführung seiner Aufgaben zu unterstützen.
\begin{enumerate}
\item Der Verein hat ordentliche, Ehren- und Fördermitglieder.
\item Ordentliche Mitglieder können ausschließlich natürliche Personen werden. Fördermitglied kann jede natürliche oder juristische Person werden, Fördermitglieder haben jedoch kein Stimmrecht bei Abstimmungen oder auf Mitgliederversammlungen.
\item Die Mitgliederversammlung kann Personen, die sich durch besondere Verdienste im Sinne des Vereins oder die von ihm verfolgten satzungsgemäßen Zwecke hervorgetan haben, zu Ehrenmitgliedern ernennen. Ehrenmitglieder haben alle Rechte eines ordentlichen Mitglieds. Sie sind von Beitragsleistungen befreit.
\item Die Beitrittserklärung erfolgt in Textform gegenüber dem Vorstand. Über die Annahme der Beitrittserklärung entscheidet der Vorstand.
\item  Gegen den ablehnenden Bescheid des Vorstands kann der Antragsteller Beschwerde einlegen, die binnen eines Monats ab Zugang der Ablehnung schriftlich beim Vorstand einzureichen ist. Über die Beschwerde entscheidet die Mitgliederversammlung.
\item Die Mitgliedschaft beginnt nach positivem Aufnahmebescheid und mit Eingang des ersten Mitgliedsbeitrags.
\end{enumerate}


\section{Beendigung der Mitgliedschaft}

Die Mitgliedschaft endet:

\begin{enumerate}
\item bei natürlichen Personen mit deren Tod.
\item nach Austrittserklärung eines Mitglieds. Die Austrittserklärung muss in Textform und gegenüber dem Vorstand mit einer Frist von drei Monaten zum Quartalsende eingereicht werden.
\item  bei Mitgliedern, die sich nach schriftlicher Mahnung mehr als sechs Monate mit Mitgliedsbeiträgen im Verzug befinden, auf Beschluss des Vorstandes.
\item durch Ausschluss aus dem Verein durch Beschluss des Vorstands, wenn das Mitglied gegen die Satzungsbestimmungen, die sich daraus ergebenden Pflichten oder in sonstiger Weise gegen die Interessen des Vereins verstößt. Das ausgeschlossene Mitglied kann innerhalb eines Monats nach Zugang des Beschlusses Einspruch einlegen und die nächste Mitgliederversammlung anrufen, von der die Gültigkeit des Ausschlusses mit Dreiviertelmehrheit der anwesenden Mitglieder bestätigt oder der Ausschluss rückgängig gemacht werden kann. Vom Zeitpunkt des Einspruchs bis zur Entscheidung über den Ausschluss ruht die Mitgliedschaft.

\end{enumerate}

\section{Mitgliedsbeiträge}
\begin{enumerate}
\item Der Verein erhebt einen regelmäßigen Mitgliedsbeitrag, Näheres regelt eine von der
Mitgliederversammlung zu beschließende Beitragsordnung.
\item Bei Beendigung der Mitgliedschaft, gleich aus welchem Grund, erlöschen alle Ansprüche
aus dem Mitgliedsverhältnis. Eine Rückerstattung von Beiträgen, Spenden oder sonstigen Unterstützungsleistungen ist grundsätzlich ausgeschlossen. Der Anspruch des Vereins auf offene Beitragsforderungen bleibt hiervon unberührt.
\item  Bei offenen Rückständen, deren Fälligkeitsdatum um zwei Monate überschritten wurde, ruht die Mitgliedschaft bis zur Begleichung aller offenen Rückstände.

\end{enumerate}

\section{Organe des Vereins}
Die Organe des Vereins sind:
\begin{enumerate}
\item die Mitgliederversammlung 
\item der Vorstand
\end{enumerate}

\section{Der Vorstand}
\begin{enumerate}
\item
Der Vorstand besteht aus
\begin{enumerate}
\item dem Vorsitzenden,
\item dem stellvertretenden Vorsitzenden und
\item dem Schatzmeister
\end{enumerate}
und ist Vorstand im Sinne des § 26 BGB.
\item  Die Vorstandsmitglieder werden für eine Dauer von zwei Jahren (Amtsperiode) gewählt.
Sie bleiben bis zur satzungsgemäßen Bestellung des nachfolgenden Vorstand im Amt. Wiederwahl ist zulässig. Bei Beendigung der Vereinsmitgliedschaft endet automatisch die Vorstandszugehörigkeit.
\item Scheidet ein Vorstandsmitglied während der Amtsperiode aus, kann der Vorstand für
die restliche Amtsdauer des ausgeschiedenen Mitglieds ein Ersatzmitglied berufen.
\item  Der Verein wird gerichtlich und außergerichtlich durch mindestens zwei Vorstandsmitglieder vertreten. Darüber hinaus kann der Vorstand einstimmig beschließen, Einzelvollmachten für konkrete Aufgaben an ordentliche Mitglieder zu erteilen.
\item Der Vorstand fasst seine Beschluüsse in öffentlichen Vorstandssitzungen. Bei Bedarf,
beispielsweise beim Umgang mit personenbezogenen Daten, Ausschlüssen und Ähnlichem,
können Tagesordnungspunkte im nichtöffentlichen Teil behandelt werden. 
\item Sitzungen werden vom Vorstand schriftlich oder mündlich mindestens zwei Tage vorher einberufen. Der Vorstand ist beschlussfähig, wenn mindestens die Hälfte seiner Mitglieder auf der Sitzung anwesend ist. Die Sitzungen und Beschlüsse des Vorstandes sind schriftlich niederzulegen und von allen anwesenden Vorstandsmitgliedern zu unterzeichnen.
\item Beschlüsse des Vorstands werden mit einfacher Mehrheit gefasst. Bei Stimmengleichheit entscheidet die Stimme des Vorsitzenden.
\item Beschlusse können auch außerhalb von Vorstandssitzungen schriftlich oder fernmündlich
gefasst werden, sofern kein Vorstandsmitglied widerspricht.
\item Der Vorstand nimmt regelmäßig folgende Aufgaben wahr:
\begin{enumerate}
\item Führung der Geschäfte
\item Entscheidung über die Aufnahme neuer Mitglieder
\item Einberufung der Mitgliederversammlung sowie Festsetzung der Tagesordnung
\item Erstellung der Rechenschaftsberichte
\item Ausführung der Beschlüsse der Jahreshaupt-/Mitgliedersammlung
\item Verwaltung des Vereinsvermögens
\end{enumerate}
\end{enumerate}


\section{Die Mitgliederversammlung}
\begin{enumerate}
\item Die ordentliche Mitgliederversammlung findet einmal jährlich statt.
\item Der Vorstand hat eine außerordentliche Mitgliederversammlung unverzüglich und unter genauer Angabe von Gründen einzuberufen, wenn es das Interesse des Vereins erfordert oder wenn mindestens 10 Prozent der Mitglieder dies schriftlich unter Angabe des Zwecks und der Gründe vom Vorstand verlangen.
\item Die Einladung zur Mitgliederversammlung erfolgt in Textform, insbesondere per E-Mail, mit einer Frist von zwei Wochen. Die Einladung gilt als bewirkt, wenn die E-Mail fristgerecht abgesendet wurde. Ist keine E-Mail-Adresse bekannt, gilt sie als bewirkt, wenn sie fristgerecht in schriftlicher Form per Post an die bekannte Adresse des Mitglieds abgesendet wurde.
\item Ein Antrag an die Mitgliederversammlung gilt als fristgemäß eingereicht, wenn er zwei Tage vor Beginn der Mitgliederversammlung beim Vorstand eingegangen ist.
\item Die Leitung der Versammlung hat ein Mitglied des Vorstands oder ein von der Mitgliederversammlung bestimmter Versammlungsleiter.
\item Stimmberechtigt ist jedes anwesende ordentliche Mitglied, dessen Mitgliedschaft nicht ruht. Fördermitglieder haben kein Stimmrecht. Jedes stimmberechtigte Vereinsmitglied hat das gleiche Stimmgewicht.
\item  Beschlüsse der Mitgliederversammlung werden mit einfacher Mehrheit gefasst. Die Beschlüsse der Mitgliederversammlung werden in einem Protokoll niedergelegt und vom Versammlungsleiter und dem Protokollführer unterzeichnet.
\item Abstimmungen müssen geheim erfolgen, wenn mindestens ein stimmberechtigtes Mitglied dies fordert.
\item Der Mitgliederversammlung obliegen:
\begin{enumerate}
\item Beschlussfassung über alle den Verein betreffenden Angelegenheiten von grundsätzlicher Bedeutung,
\item Entscheidung über fristgemäß eingebrachte Anträge,
\item Entgegennahme des Jahresbericht des Vorstands,
\item Entlastung des Vorstands,
\item Wahl der Vorstandsmitglieder,
\item Wahl der Kassenprüfer,
\item Beschlussfassung über Satzungs- und Satzungszweckänderungen,
\item Festsetzung der Mitgliedsbeiträge,
\item Entscheidung über Einsprüche von Ausschlüssen aus dem Verein,
\item Die Auflösung des Vereins
\end{enumerate}
\item  Die Mitgliederversammlung bestimmt mindestens zwei Kassenprüfer, üblicherweise
nach der Wahl eines Vorstands und ebenfalls für eine Dauer von zwei Jahren. Die Mitgliederversammlung kann eine abweichene Anzahl und Dauer bestimmen. Die Kassenprüfer kontrollieren die Arbeit des Schatzmeisters und berichten der Mitgliederversammlung.
\end{enumerate}

\section{Auflösung des Vereins}
\begin{enumerate}
\item Über die Auflösung des Vereines entscheidet eine Mitgliederversammlung, die eigens zu diesem Zweck einberufen wird. Die Auflösung gilt als beschlossen, wenn dreiviertel der abgegebenen Stimmen dafür stimmen.
\item  Bei Auflösung oder Aufhebung der Körperschaft oder bei Wegfall steuerbegünstigter Zwecke fällt das Vermögen an den Entropia e.V., geführt unter der Nummer 102736 im Vereinsregister des Amtsgerichts Mannheim, welche es unmittelbar für gemeinnützige Zwecke verwenden darf.
\end{enumerate}


\section{Schlussbestimmung}
\begin{enumerate}[I.]
\item Der Vorstand ist befugt, redaktionelle Änderungen an dieser Satzung durchzuführen,
sofern sie einer Auflage des Registergerichts oder einer Behörde entsprechen muss. Über diese Änderungen müssen die Mitglieder unverzüglich informiert werden.
\end{enumerate}
\end{document}
